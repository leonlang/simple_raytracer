\documentclass{article}
\usepackage{amsmath}
\usepackage{amssymb}
\usepackage{bm}

\begin{document}
	
	\section*{Phong Illumination Model}

	
	\subsection*{Diffuse Reflection}
	\begin{itemize}
		\item Diffuse reflection is based on Lambert's cosine law, which states that the intensity of light is proportional to the cosine of the angle between the light direction \textbf{l} and the surface normal \textbf{n}.
		\item The intensity of light is higher if the angle is sharper.
		\item The dot product $\bm{n} \cdot \bm{l}$ represents this cosine value, and I use max to ensure it is non-negative.
		\item $\bm{p} \cdot \bm{L}_{\text{light}}$ represents the final color of the object with light.
	\end{itemize}
	\begin{align*}
		\bm{diffuse} &= r_{\text{View}} \cdot \bm{p} \cdot \bm{L}_{\text{light}} \cdot \max(\bm{n} \cdot \bm{l}, 0.0)
	\end{align*}
	
	\subsection*{Ambient Reflection}
	\begin{itemize}
		\item Ambient reflection represents the constant illumination of the object by the environment.
		\item It is usually a small constant value added to ensure that objects are visible even when not directly lit.
		\item Higher ambient strength means the entire object brightens up more by the same amount.
	\end{itemize}
	\begin{align*}
		\bm{ambient} &= \frac{1}{\pi} \cdot \textit{ambientStrength} \cdot \bm{p} \cdot \bm{L}_{\text{light}}
	\end{align*}
	
	\subsection*{Specular Reflection}
	\begin{itemize}
		\item Specular reflection represents the mirror-like reflection of light sources on shiny surfaces.
		\item It does not use the object color ($\bm{p}$) because specular highlights are typically the color of the light source.
		\item Higher shininess ($m$) means a smaller specular highlight.
	\end{itemize}
	\begin{itemize}
		\item It is based on the dot product between the view direction \textbf{v} and the reflection direction \textbf{r}, raised to the power of the shininess factor ($m$).
		\item $s$ is the specular strength. Smaller specular strength means less intensity.
	\end{itemize}
	\begin{align*}
		\bm{specular} &= \bm{L}_{\text{light}} \cdot s \cdot \max(\bm{n} \cdot \bm{l}, 0.0) \cdot \left(\max(\bm{r} \cdot \bm{v}, 0.0)\right)^m
	\end{align*}
	
	\subsection*{Final Color}
	\text{Combine the three components to get the final color:}
	\begin{align*}
		\bm{finalColor} &= \bm{diffuse} + \bm{specular} + \bm{ambient}
	\end{align*}
	
\end{document}
